	\documentclass[fleqn, 12pt]{article}
	\usepackage[latin1]{inputenc}
	\usepackage[ngerman]{babel}
	\usepackage[a4paper,text={155mm,220mm},centering,headsep=10mm,footskip=15mm]{geometry}
	\usepackage{graphicx}
	\usepackage{amssymb}
	\usepackage{graphicx}
	\usepackage{multirow}
	\usepackage{multicol}
	\usepackage{float}
	\usepackage{pdfpages}
	\usepackage{todonotes}
	\usepackage{pst-plot}
	\usepackage{pstricks}
	\usepackage{pstricks-add}
	\usepackage{pst-node}
	\usepackage{pst-grad,multido} 
	\title{}
	\date{}
	\author{}
	\usepackage{array}
	\usepackage{amsmath}
	\usepackage{fancyhdr}
	\usepackage{enumitem}	
	\usepackage{blindtext}
	
	% Listings
	\usepackage{listings}
	\renewcommand{\lstlistlistingname}{Liste der Quellcode Ausschnitte}
	\renewcommand{\lstlistingname}{Quellcode Ausschnitt}
		\definecolor{lsgreen}{rgb}{0,.5,0}
		\definecolor{lsred}{rgb}{.7,0,0}
		\definecolor{lsorange}{rgb}{.9,.5,0}
		\definecolor{lsgray}{rgb}{.5,.5,.5}
		\lstset{
			frame=tb,
			aboveskip=10mm,
			belowskip=10mm,
			showstringspaces=false,
			columns=flexible,
			captionpos=b,
			basicstyle={\normalsize\ttfamily},
			numbers=left,
			numberstyle=\tiny\color{lsgray},
			keywordstyle=\color{blue},
			commentstyle=\color{lsorange},
			stringstyle=\color{lsgreen},
			breaklines=true,
			breakatwhitespace=true,
			rulecolor=\color{lsgray},
			xleftmargin=7mm,
			tabsize=3
		}
	
	% Needed for proper display of links in bibliography
	\usepackage{url}	
	
	% Hyperref
	\usepackage[
	pdfauthor={Laura Anger, Timo \dots, Lukas Kolhagen},
	pdftitle={BVA Bewegungsanalyse},
	pdftoolbar=true,	
	colorlinks=true,
	linkcolor=blue,
	citecolor=blue,
	urlcolor=blue,
	linktocpage=true
	]{hyperref}
	\usepackage{bookmark}
	\bookmarksetup{
	numbered
	}
	\urlstyle{same}
	
	\pagestyle{fancy}
	\fancyhf{}
	\fancyhead[LO]{BVA}
	\fancyhead[CO]{Bewegungsanalyse in einer Videosequenz}
	\fancyhead[RO]{\thepage}
	\renewcommand{\headrulewidth}{0.5pt}
	\newcommand{\Absatzbox}[1]{\parbox[0pt][2em][c]{0cm}{}}
	
	% Itemize symbol
	\renewcommand{\labelitemi}{$\triangleright$}
	
	% Colors
	\definecolor{yellow}{rgb}{.95,.85,0}
	\definecolor{green}{rgb}{0,.8,0}
	\definecolor{blue}{rgb}{0,0,.8}
	\definecolor{red}{rgb}{.8,0,0}
	\definecolor{grey}{rgb}{.4,.4,.4}
	\definecolor{orange}{rgb}{.9,.5,0}
	
	% Macro for typesetting C++
	\newcommand{\CC}{C\nolinebreak\hspace{-.05em}\raisebox{.4ex}{\tiny\bf +}\nolinebreak\hspace{-.10em}\raisebox{.4ex}{\tiny\bf +}}
\def\CC{{C\nolinebreak[4]\hspace{-.05em}\raisebox{.4ex}{\tiny\bf ++}}}

\begin{document}
\thispagestyle{empty}
	%\sffamily
			\begin{center}
			\includegraphics[width=.35\textwidth]{logo_TH}\\[20ex]
			{\Huge\textbf{Projektdokumentation}}\\[8ex]
			\rule{.8\textwidth}{.2pt}
			{\Large Bewegungsanalyse einer Videosequenz\\[1ex] mit dem Ansatz des Papers
			von Aach und Kunz}\\
			\rule{.8\textwidth}{.2pt}\\[10ex]
			von\\[2ex]
			\begin{tabular}{ll}
			Laura Anger &(Matrikelnr. 11086356)\\ 
			Timo Breuer &(Matrikelnr. XXXXXXXX)\\ 
			Lukas Kolhagen &(Matrikelnr. 11084355)\\
			\end{tabular}\\[10ex]
			Durchgef�hrt im\\ \textbf{Master Medientechnologie}\\ im\\ 
			\textbf{Sommersemester 2016}\\			
			\end{center}
			\vfill
			\begin{flushleft}
			{\bf Betreuer:}\\
			Prof. Dr. Dietmar Kunz\\
			Institut f�r Medien- und Phototechnik
			\end{flushleft}
	\newpage
	\tableofcontents
	\newpage
	\section{Einleitung}\todo[inline]{Lukas}
	Diese Ausarbeitung ist Teil der Abschlussprojekt-Dokumentation im Modul "`Weiterf�hrende Themen der Bildverarbeitung"' im Master Medientechnologie an der Technischen Hochschule K�ln.
	
	Das Projekt besch�ftigte sich mit der Bewegungsanalyse einer Videosequenz mit dem Ansatz des Papers von Aach und Kunz\textsuperscript{\cite{aach1998bayesian}}. Es wurde bearbeitet von Laura Anger, Timo Breuer und Lukas Kolhagen.
	\subsection{Ansatz im Paper von Aach und Kunz}
	Die Grundlage f�r das vorliegende Projekt bildet das Paper "`Bayesian motion estimation for temporally recursive noise reduction in X-ray fluoroscopy"'. Dieses besch�ftigt sich mit der Entwicklung einer robusten Methode zur Bewegungssch�tzung f�r die speziellen Anforderungen der stark rauschenden Aufnahmen einer R�ntgen-Fluoroskopie.\\
	Der Ansatz beruht auf der Modellierung drei essenzieller Faktoren:
	\begin{description}
		\item [Datenterm:] Unterschied der Grauwerte zweier aufeinander folgender Bildern.
		\item [�rtliche Koh�renz:] Au�er an Randbereichen von Objekten, bewegen sich Nachbarschaften meist in die gleiche Richtung.
		\item [Zeitliche Koh�renz:] Bewegungen verlaufen normalerweise kontinuierlich, sodass sich ein Bildblock zwischen zwei Bildern wahrscheinlich in dieselbe Richtung weiterbewegt oder die Richtung nur gering �ndert.
	\end{description}
	Eine genaue Beschreibung dieser Faktoren erfolgt in \ref{sec:costFunc}.\\
	
	\subsection{Projektziel}
	Die Zielsetzung f�r das Projekt "`Bewegungsanalyse einer Videosequenz"' war eine �ber\-tra\-gung des Ansatzes von R�ntgenbildern auf normale Videosequenzen. Infolge dessen war eine Vernachl�ssigung der speziellen Anforderung des Bildrauschens m�glich, da R�ntgenbilder -- insbesondere als Teil einer Fluoroskopie -- zum Schutz des Patienten und des medizinischen Personals nur sehr geringe R�ntgendosen enthalten d�rfen und deshalb ein extrem schlechtes Signal-zu-Rauschverh�ltnis aufweisen. Diese Problematik besteht bei normalen Videosequenzen nicht, weshalb f�r die Untersuchung von vergleichsweise geringem und etwa gleich verteiltem Rauschen ausgegangen werden konnte.
	\section{Verfahren}
		\subsection{Bewegungssch�tzung}\todo[inline]{Laura}
		\newpage %Muss sp�ter entfernt werden
		\subsection{Programmablauf}\todo[inline]{Lukas}
			\begin{figure}[h!]
			\centering
					\begin{psmatrix}[rowsep=0.3,colsep=0.4]
						\rnode{run}{\psframebox[fillstyle=solid,fillcolor=black]{\footnotesize\color{white}public void run(ImageProcessor ip)}}\\
						\rnode{readStack}{\psframebox[linearc=0.05,cornersize=absolute,fillstyle=solid,fillcolor=black!10]{\footnotesize Read data from stack}}\\
						\rnode{initialize}{\psframebox[linearc=0.05,cornersize=absolute,fillstyle=solid,fillcolor=black!10]{\footnotesize Initialize first motion vector field with zero vectors}}\\
						\rnode{forI}{\psframebox[linearc=0.05,cornersize=absolute,fillstyle=solid,fillcolor=black!10]{\footnotesize\texttt{$i=0$}}}\\
						\dianode[fillstyle=solid,fillcolor=red!30]{condStackSize}{\footnotesize$i<$ \texttt{stack size}?}\\[3mm]
						\dianode[fillstyle=solid,fillcolor=red!30]{frameNotZero}{\footnotesize$i\not=$ \texttt{0}?}\\[3mm]
						\rnode{curFrame}{\psframebox[linearc=0.05,cornersize=absolute,fillstyle=solid,fillcolor=black!10]{\footnotesize Set current frame to i}}\\
						\rnode{forJ}{\psframebox[linearc=0.05,cornersize=absolute,fillstyle=solid,fillcolor=black!10]{\footnotesize\texttt{$j=0$}}}\\
						\dianode[fillstyle=solid,fillcolor=red!30]{condIter}{\footnotesize$j<$ \texttt{iterations}?}\\[3mm]
						\rnode{Vn}{\psframebox[linearc=0.05,cornersize=absolute,fillstyle=solid,fillcolor=black!10]{\footnotesize Set current vector field as previous}}\\
						\rnode{forK}{\psframebox[linearc=0.05,cornersize=absolute,fillstyle=solid,fillcolor=black!10]{\footnotesize\texttt{$k=0$}}}\\
						\dianode[fillstyle=solid,fillcolor=red!30]{condNumBlocks}{\footnotesize$k<$ \texttt{number of blocks}?}\\[3mm]
						\rnode{calcAlt}{\psframebox[linearc=0.05,cornersize=absolute,fillstyle=solid,fillcolor=black!10]{\footnotesize Calculate 14 alternatives for block k}}\\
						\rnode{minimizeCost}{\psframebox[linearc=0.05,cornersize=absolute,fillstyle=solid,fillcolor=black!10]{\footnotesize Minimize cost function for alternatives}}\\
						\rnode{addVec}{\psframebox[linearc=0.05,cornersize=absolute,fillstyle=solid,fillcolor=black!10]{\footnotesize Add vector with minimal cost to motion vector field}}\\
						\rnode{updateProgress}{\psframebox[linearc=0.05,cornersize=absolute,fillstyle=solid,fillcolor=black!10]{\footnotesize Update plug-in progress}}\\
						\rnode{genMVF}{\psframebox[linearc=0.05,cornersize=absolute,fillstyle=solid,fillcolor=black!10]{\footnotesize Generate motion vector field visualization for current frame}}\\
						\rnode{addToOutStack}{\psframebox[linearc=0.05,cornersize=absolute,fillstyle=solid,fillcolor=black!10]{\footnotesize Add motion vector field to output stack}}\\
						\rnode{outWindow}{\psframebox[linearc=0.05,cornersize=absolute,fillstyle=solid,fillcolor=black!10]{\footnotesize Generate and display output window}}\\
					\end{psmatrix}
					\psset{arrows=->,nodesep=0pt}
					\ncline{run}{readStack}
					\ncline{readStack}{initialize}
					\ncline{initialize}{forI}
					\ncline{forI}{condStackSize}
					\ncline{condStackSize}{frameNotZero}\nbput{\footnotesize\color{green}True}
					\ncangle[angle=0, linearc=.1]{frameNotZero}{genMVF}\nbput[npos=.15]{\footnotesize\color{red}False}
					\ncline{frameNotZero}{curFrame}\nbput{\footnotesize\color{green}True}
					\ncline{curFrame}{forJ}
					\ncline{forJ}{condIter}					
					\ncline{condIter}{Vn}\nbput{\footnotesize\color{green}True}
					\ncline{Vn}{forK}
					\ncline{forK}{condNumBlocks}
					\ncline{condNumBlocks}{calcAlt}\nbput{\footnotesize\color{green}True}
					\ncline{calcAlt}{minimizeCost}
					\ncline{minimizeCost}{addVec}
					\ncline{addVec}{updateProgress}
					\ncangle[arm=3cm, angle=180, linearc=.1]{condIter}{updateProgress}\nbput[npos=.25]{\footnotesize\color{red}False}
					\ncangle[arm=1.8cm, angle=0, linearc=.1]{condNumBlocks}{condIter}\nbput[npos=.25]{\footnotesize\color{red}False}
					\ncline{updateProgress}{genMVF}
					\ncline{genMVF}{addToOutStack}
					\ncline{addToOutStack}{outWindow}
					\ncangle[arm=2.8cm, angle=180, linearc=.1]{condStackSize}{outWindow}\nbput[npos=.2]{\footnotesize\color{red}False}
			\caption{Programmablauf}
			\label{fig:programmAblauf}
		\end{figure}
		
		\subsection{Kostenfunktion}\label{sec:costFunc}\todo[inline]{Alle}
			\subsubsection{Datenterm}\todo[inline]{Lukas}
			\subsubsection{�rtliche Koh�renz}\todo[inline]{Timo}
			\subsubsection{Zeitliche Koh�renz}\todo[inline]{Lukas}
		\subsection{Visualisierung}\todo[inline]{Laura}
	\section{Auswertung}\todo[inline]{Laura \& Timo}
		\subsection{Testmaterial}\todo[inline]{Laura}
		\subsection{Analyse der Kostenfunktion}\todo[inline]{Laura}
		\subsection{Einschwingverhalten}
			\subsubsection{Bewegungsvektorfelder}\todo[inline]{Timo}
			\subsubsection{Innerhalb eines Bildes}\todo[inline]{Laura}
		\subsection{Parameter der Regularisierungsterme}\todo[inline]{Timo}
	\section{Zusammenfassung}\todo[inline]{Lukas}
	\section{Arbeitsaufteilung der Dokumentation}
	\newpage
	\bibliographystyle{plain}
	\bibliography{ABV_Bewegungsanalyse_LaTeX}
\end{document}
		